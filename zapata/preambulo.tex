\documentclass[12pt,addpoints]{article}
\usepackage[utf8]{inputenc}
\usepackage[table, dvipsnames]{xcolor}
%\usepackage[usenames]{color}
\usepackage{stmaryrd}
%\usepackage{array}
\usepackage{xfp}
\usepackage{fp}
\usepackage{pgf}
\usepackage{graphicx}
\usepackage{subfigure}
\usepackage{mathtools}
\usepackage[natbibapa]{apacite}
\bibliographystyle{apacite}
\graphicspath{ {images/} }
\usepackage{vmargin}
\usepackage{amsmath}
\usepackage{circuitikz}
\usepackage{tikz}
\usepackage{tocloft}
\usetikzlibrary{calc}
\usetikzlibrary{arrows}

\usepackage{pgfplots}
\pgfplotsset{compat=1.10}
\usepgfplotslibrary{fillbetween}
\usetikzlibrary{patterns}


\usepackage{units}
\usepackage{setspace}
\usepackage{multicol}
\usepackage{multirow}
\usepackage{colortbl}
\usepackage{array}
\usepackage{booktabs}
\usepackage{caption}
\usepackage{amssymb}
\usepackage{amsfonts}
\usepackage{amsthm}
\usepackage{amsmath,yhmath}
\usepackage{geometry}
%\usepackage{subcaption}
\usepackage{graphicx}
\usepackage[export]{adjustbox}
\usepackage[framemethod=tikz]{mdframed}
\usepackage{lipsum}
\usepackage{tcolorbox}
\usepackage{tocloft}
\usepackage{fancyhdr}
\usepackage{float}
%\usepackage[colorlinks,citecolor=red]{hyperref}
%\usepackage{}
%\newtcolorbox{mybox2}{colback=red!5!white,colframe=red!75!black,width=0.85\textwidth}
\newtcolorbox{mybox2}[1]{colback=gray!5!white,colframe=cyan!75!black,fonttitle=\bfseries,title=#1}

\newtcolorbox{mybox3}[1]{colback=gray!5!white,colframe=Maroon!75!black,fonttitle=\bfseries,title=#1}

\newtcolorbox{mybox4}[1]{colback=gray!5!white,colframe=LimeGreen!75!black,fonttitle=\bfseries,title=#1}


\newcommand{\listequationsname}{\Large Lista de ecuaciones}
\newlistof{myequations}{equ}{\listequationsname}
\newcommand{\myequations}[1]{%
\addcontentsline{equ}{myequations}{Ecuación\hspace{0.3em}\protect\numberline{\theequation}#1}\par}
\setlength{\cftmyequationsnumwidth}{1.75em}
\setlength{\cftmyequationsindent}{1.5em}
\addtocontents{equ}{~\hfill\textbf{Página}\par}


\definecolor{mycolor}{rgb}{0.122, 0.435, 0.698}
%\captionsetup[table]{name=Tabla}
%\usepackage{bigstrut}
\setpapersize{A4}
\setmargins{2.5cm}       % margen izquierdo
{1.5cm}                        % margen superior
{16.5cm}                      % anchura del texto
{23.42cm}                    % altura del texto
{10pt}                           % altura de los encabezados
{1cm}                           % espacio entre el texto y los encabezados
{0pt}                             % altura del pie de página
{2cm}                           % espacio entre el texto y el pie de página
\usepackage{diagbox}
\usepackage{slashbox}
\usepackage{url}
\usepackage[framemethod=TikZ]{mdframed}
%Theorem
\newcounter{theo}[section] \setcounter{theo}{0}
\renewcommand{\thetheo}{\arabic{section}.\arabic{theo}}
\newenvironment{theo}[2][]{%
\refstepcounter{theo}%
\ifstrempty{#1}%
{\mdfsetup{%
frametitle={%
\tikz[baseline=(current bounding box.east),outer sep=0pt]
\node[anchor=east,rectangle,fill=blue!20]
{\strut Theorem~\thetheo};}}
}%
{\mdfsetup{%
frametitle={%
\tikz[baseline=(current bounding box.east),outer sep=0pt]
\node[anchor=east,rectangle,fill=blue!20]
{\strut  ~#1};}}%%%%%%%%%Theorem~\thetheo:
}%
\mdfsetup{innertopmargin=10pt,linecolor=blue!20,%
linewidth=2pt,topline=true,%
frametitleaboveskip=\dimexpr-\ht\strutbox\relax
}
\begin{mdframed}[]\relax%
\label{#2}}{\end{mdframed}}
%%%%%%%%%%%%%%%%%%%%%%%%%%%%%%

\newtheorem{defi}{Inciso}[subsection]
\renewcommand{\listfigurename}{Lista de Figuras}
\renewcommand{\listtablename}{Lista de Tablas}
\renewcommand{\baselinestretch}{1.5}
\renewcommand{\abstractname}{Resumen}
\renewcommand{\figurename}{Figura}
\renewcommand{\contentsname}{\underline{Contenido}}
\renewcommand{\tablename}{Tabla}
\renewcommand{\cftfigfont}{Figura }
\renewcommand{\cfttabfont}{Tabla }
\addtocontents{toc}{~\hfill\textbf{Página}\par}
\addtocontents{lot}{~\hfill\textbf{Página}\par}
\addtocontents{lof}{~\hfill\textbf{Página}\par}
\providecommand{\keywords}[1]
{
  \textbf{\text{Palabras clave: }} #1
}

\title{\textbf{MEMORIA DE CALCULO ESTRUCTURAL}}
\author{VIVIENDA MULTIFAMILIAR}
\date{Junio 06, 2022}
\usepackage[hidelinks]{hyperref}

\pagestyle{fancy}
\fancyhf{}
\rhead{\includegraphics[trim={0 5cm 0 2.2cm},clip,width=27mm]{PORTADA (4).png}}
\lhead{MEMORIA DE CALCULO}
%\rfoot{Página\ \thepage\ de \numpages}
\lfoot{Realizado por: \textit{Alexis Pompilla Yábar}}
\cfoot{}
\renewcommand{\headrulewidth}{0.4pt}
\renewcommand{\footrulewidth}{0.4pt}
\rfoot{Página\ \thepage}
%\AtBeginDocument{\addtocontents{toc}{\protect\thispagestyle{empty}}} 


%%%%%%%%%%%%%%%%%%%%%%%%%%%%%%%%%%%%%%%%%%%%%%%%%%%%%%%%%%%%%%%%%%%%%%%%%%%%%%%%%%%%%%%%%%%%%%%%%%%%%%%%%%%%
%%%%%%%%%%%%%%%%%%%%%%%%                        Nuevos comandos         %%%%%%%%%%%%%%%%%%%%%%%%%%%%%%%%%%%%
\setlength{\parindent}{0pt} % sin sangría
%\renewcommand{\multirowsetup}{\centering}
%\newlength{\LL}\settowidth{\LL}{texto}

%% -------------------------------------------------------------------------------------------------------- %%
%%                                          ECUACIONES                                                      %%
%% -------------------------------------------------------------------------------------------------------- %%

\usepackage{xargs}


\newcommandx{\mycommandEC}[4][1=a, 2=b, 3=c, 4=]{%
	% Cuerpo del comando
	% Aquí puedes utilizar los argumentos #1, #2 y #3
	%\begin{align#4}
		x_{1, 2} &= \dfrac{- #2 \pm \sqrt{#2^2 - 4 \cdot #1 \cdot #3} }{2 \cdot #1} 		%\tag{A}
	%\end{align#4}
}

% Longitud de desarrollo %Varibles db, fy, f'c
\newcommandx{\Ldi}[3][1=d_b, 2=f_y, 3=f'_c]{0.08 \cdot #1 \cdot \dfrac{#2}{\sqrt{#3}}}
\newcommandx{\Ldii}[2][1=d_b, 2=f_y]{0.004 \cdot #1 \cdot #2}

% d = max(ld1, ld2)
\newcommandx{\ddd}[2][1=L_{d1}, 2=L_{d2}]{\max(#1, #2)}

% hz = d + 0.1m
\newcommandx{\hz}[1][1=d]{d + 0.1m}

% Capacidad portante neta del terreno
% Variables: σn = σt - γc·hz - γm·hs - γc·hp - S/C_piso
\newcommandx{\qn}[7][1=\sigma_t, 2=\gamma_c, 3=h_z, 4=\gamma_m, 5=h_s, 6=h_p, 7=S/C_{piso}]{#1 - #2 \cdot #3 - #4\cdot#5 - #2\cdot#6 - #7}

% Área tentativa % Variables: Pservicio, σn
\newcommandx{\AreaTentativa}[2][1=P_{servicio}, 2=\sigma_n]{\dfrac{#1}{#2}}
 
% Dimensiones de la zapata L y B
% Variables: At, c1, c2
\newcommandx{\LongitudZapL}[3][1=A_t, 2=c_1, 3=c_2]{\sqrt{#1} + \dfrac{#2 - #3}{2}} %Para L
\newcommandx{\LongitudZapB}[3][1=A_t, 2=c_1, 3=c_2]{\sqrt{#1} - \dfrac{#2 - #3}{2}} %Para B
\newcommandx{\Area}[2][1=L, 2=B]{#1 \cdot #2}

% Verificación de zapatas con cargas y momentos biaxiales
% variables: Pultima, Area(A), Mx, Vx, Iyy + My, Vy, Ixx
\newcommandx{\EcPresiones}[8][1=P_{ultima}, 2=A, 3=M_x, 4=V_x, 5=I_{yy}, 6=M_y, 7=V_y, 8=I_{xx}]{\sigma_{1,2,3,4} = \dfrac{#1}{#2} \pm \dfrac{#3\cdot#4}{#5} \pm \dfrac{#6\cdot#7}{#8}}

% Verificación de corte por punzonamiento
% Sección crítica bo, variables: c1, c2, d
\newcommandx{\SeccCriticaP}[3][1=c_1, 2=c_2, 3=d]{2(#1 + #3) + 2(#2 + #3)}
% Area tributaria, variables: c1, c2, d
\newcommandx{\AreaTributariaP}[3][1=c_1, 2=c_2, 3=d]{(#1 + #3)\cdot(#2 + #3)}

% Cortante de diseño por corte-punzonamiento
% Variables: σu, A, Ao
\newcommandx{\VuPunzonamiento}[3][1=\sigma_u, 2=A, 3=A_o]{#1 \cdot (#2 - #3)}

% Verificación cortante por punzonamiento (las 3 verificaciones)
% Variables: σ=0.85, f'c, bo, d, αs, β
\newcommandx{\VcPunzonamientoi}[4][1=f'_c, 2=b_o, 3=d, 4=\beta]{0.85 \cdot (0.17 \left( 1 + \dfrac{2}{#4} \right) \sqrt{#1}\cdot #2 \cdot #3)}

\newcommandx{\VcPunzonamientoii}[5][1=f'_c, 2=b_o, 3=d, 4=\beta, 5=\alpha_s]{0.85 \cdot ( 0.083 \left( \dfrac{#5 \cdot #3}{#2} + 2\right) \sqrt{#1}\cdot #2 \cdot #3)}

\newcommandx{\VcPunzonamientoiii}[3][1=f'_c, 2=b_o, 3=d]{0.85 \cdot (0.33 \cdot \sqrt{#1} \cdot #2 \cdot #3)}


% Cortante de diseño por flexion
% Variables: σu, B, c, d
\newcommandx{\VuFlexion}[4][1=\sigma_u, 2=B, 3=c, 4=d]{#1 \cdot #2 \cdot (#3 - #4)}

% Verificación cortante por corte-flexion
% Variables: σ=0.85, f'c, B, d
\newcommandx{\VcFlexion}[3][1=f'_c, 2=B, 3=d]{0.85 \cdot 0.17 \cdot \sqrt{#1} \cdot #2 \cdot #3}

% Cálculo de acero por flexión
% Variables: σu, Lv=, B=prof de la zapata
\newcommandx{\MomUltimoi}[3][1=\sigma_u, 2=L_v, 3=B]{\dfrac{#1 \cdot (#2)^2 \cdot #3}{2}}

% Variables: Mu, φ, B, d, f'c, %%w
\newcommandx{\MomUltimoii}[5][1=M_u, 2=\phi, 3=B, 4=d, 5=f'c]{#1 = #2 (#3)(#4^2)(#5)(w)(1 - 0.59 \cdot w)}

% Cálculo del área de acero
% Variables: w, b, d, f'c, fy
\newcommandx{\AreaAcero}[5][1=w, 2=b, 3=d, 4=f'_c, 5=f_y]{\dfrac{#1 \cdot #2 \cdot #3 \cdot #4}{#5}}

% Área de acero mínimo
% Variables: b, h
\newcommandx{\AreaAceroMin}[2][1=b, 2=h]{0.0018 \cdot #1 \cdot #2}