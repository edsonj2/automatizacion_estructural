%\setlength{\abovedisplayskip}{0pt}
%\setlength{\belowdisplayskip}{0pt}
%\setlength{\abovedisplayshortskip}{0pt}
%\setlength{\belowdisplayshortskip}{0pt}
\maketitle
\begin{figure}[h!]
    \centering
    \includegraphics[scale=0.75]{IMAGENES/r2.png}
    \label{fig:my_label}
\end{figure}
\thispagestyle{empty}
\newpage

\clearpage                       % Otherwise \pagestyle affects the previous page.
{                                % Enclosed in braces so that re-definition is temporary.
  \pagestyle{empty}              % Removes numbers from middle pages.
  \fancypagestyle{plain}         % Re-definition removes numbers from first page.
  {
    \fancyhf{}%                       % Clear all header and footer fields.
    \renewcommand{\headrulewidth}{0pt}% Clear rules (remove these two lines if not desired).
    \renewcommand{\footrulewidth}{0pt}%
  }
    \begin{spacing}{1.35}
    \tableofcontents
  \end{spacing}
  \thispagestyle{empty}  
  \listoffigures
\newpage
\listoftables
  \thispagestyle{empty} 
% Removes numbers from last page.
}


%\listofmyequations
%\header{Realizado por: Alexis Pompilla Yábar}{}{}
\newpage


\input{12}


\newpage
\section{Diseño de la Cimentación}

\subsection{Diseño de Zapata Aislada}

%% ------------------------------------------------------------------------------------
%%                      DEFINICIÓN DE VARIABLES                                         
%% ------------------------------------------------------------------------------------
\FPset\plosa{0.3} %peso de losa por m2
\FPset\alosa{3.2}
\FPset\wpt{0.1}
\FPset\wlosa{0.2}
\FPset\bv{0.3}
\FPset\h{0.5}

% CARGAS
\FPset\Pm{130} % Carga muerta
\FPset\Pv{70} % Carga viva
\FPset\Mmx{10} % Momento por carga muerta en x
\FPset\Mmy{2} % Momento por carga muerta en y
\FPset\Mvx{6} % Momento por carga viva en x
\FPset\Mvy{1} % Momento por carga viva en y
\FPset\Msx{15} % Momento por carga sísmica en la dirección x
\FPset\Msy{13} % Momento por carga sísmica en la dirección y
\FPset\Psx{10} % Carga sísmica vertical en la dirección x
\FPset\Psy{9} % Carga sísmica vertical en la dirección y

\FPset\q{30} % Resistencia del terreno 30 ton/m2

% Dimensiones de la zapata
\FPset\B{2.60} % B = 2.60m
\FPset\L{3.00} % L = 3.00m


% Cálculos
\FPeval{\P}{round(\Pm+\Pv,2)} % Pm + Pv
\FPeval{\At}{round(\P * 1.05 / 27,2)} % Área tentativa
\FPeval{\A}{round(\B * \L,1)} % Área
\FPeval{\M}{round(\Mmx+\Mvx,2)} % Mmx + Mvx
\FPeval{\Sigmax}{round(\P * 1.05/ (\B * \L) + 6*\M / (\B * \L^2),2)} % Sigma en dirección x

% Aumentar las dimensiones de la zapata
\FPeval{\Ba}{round(\B + 0.1,2)} % B = 2.60m + 0.1m
\FPeval{\La}{round(\L + 0.1,2)} % B = 3.00m + 0.1m

\FPeval{\Sigmaxa}{round(\P * 1.05/ (\Ba * \La) + 6*\M / (\Ba * \La^2) + 6*(\Mmy + \Mvy)/(\La* \Ba^2),2)} % Sigma en dirección x
\FPeval{\Px}{round(\Pm+\Pv + \Psx,2)} % Pm + Pv + Psx
\FPeval{\Sigmaxsa}{round(\Px * 1.05/ (\Ba * \La) + 6*(\M + \Msx) / (\Ba * \La^2) + 6*(\Mmy + \Mvy)/(\La* \Ba^2),2)} % Sigma con sismo en dirección x 

\FPeval{\Py}{round(\Pm+\Pv + \Psy,2)} % Pm + Pv + Psy
\FPeval{\Sigmaysa}{round(\Py * 1.05/ (\Ba * \La) + 6*(\M ) / (\Ba * \La^2) + 6*(\Mmy + \Mvy + \Msy)/(\La* \Ba^2),2)} % Sigma con sismo en dirección y 


%% ------------------------------------------------------------------------------------
%%                      EVALUACIÓN DE VARIABLES                                         
%% ------------------------------------------------------------------------------------
\FPeval{\wviga}{round((\plosa+\wpt)*\alosa+2.4*\bv*\h/10000,2)}
\FPeval{\wvigav}{round((\wlosa)*\alosa,2)}
\FPeval{\wvigau}{round(1.25*(\wviga+\wvigav),2)}

%% ------------------------------------------------------------------------------------

\subsubsection{Datos para el diseño de una zapata aislada con carga y momentos}

\begin{table}[h!]
    \centering
    \rowcolors{1}{}{gray!20}
    \begin{tabular}{lcl} %\toprule
        Dimensiones de la columna               &:& $C_1        = 0.40$m \quad $C_2 = 0.80$m\\
        Profundidad de cimentación              &:& $D_f        = 1.70$m\\
        Altura de piso terminado                &:& $h_p        = 0.10$m\\
        Resistencia a compresión del concreto   &:& $f'_c       = 210$ kg/cm$^2$ \\
        Resistencia a la fluencia del acero     &:& $f_y        = 4200$ kg/cm$^2$ \\
        Peso específico del relleno             &:& $\gamma_m   = 2.1$ ton/m$^3$ \\
        Peso específico del concreto            &:& $\gamma_c   = 2.4$ ton/m$^3$ \\
        Sobrecarga de piso                      &:& $S/C_{piso}$= 500 kg/m$^2$ \\ 
        Capacidad portante del terreno          &:& $\sigma_t$  = 3 kg/cm$^2$  \\ %= 30 ton/m$^2$
    \end{tabular}
\end{table}

\textbf{Cargas:}


\begin{table}[h!]
    \centering
    \begin{tabular}{cccc} \toprule
         & Carga en la dirección Z & Momento en la dirección X & Momento en la dirección Y  \\ 
         & $F_z$ & $M_x $ & $M_y$ \\ \midrule
        $P_m$ & 130& 10 & 2 \\
        $P_v$ & 70 & 6 & 1 \\
        $S_x$ & 10 & 15 & 0 \\
        $S_y$ & 9 & 0 & 13 \\
        $V_x$ & 180 & 16 & 11 \\
        $V_y$ & 180 & 16 & 11 \\
        $P_p$ & 180 & 16 & 11 \\\bottomrule
    \end{tabular}
    \caption{Cargas y momentos para el diseño}
    \label{tab:my_label}
\end{table}

\begin{table}[h!]
    \centering
    \begin{tabular}{lll}
        $P_m$ &=&   Carga muerta\\
        $P_v$ &=&   Carga viva\\
        $S_x$ &=&   Carga sísmica debido al sismo en la dirección x\\
        $S_y$ &=&   Carga sísmica debido al sismo en la dirección y\\
        $V_x$ &=&   Carga por viento en la dirección x\\
        $V_y$ &=&   Carga por viento en la dirección y\\
        $P_p$ &=&   Peso propio\\
    \end{tabular}
\end{table}

\subsubsection{Dimensionamiento en altura}
Longitud de desarrollo

\begin{align}
	% Longitud de desarrollo
	% \Ldi[db][fy][f'c]
	% \Ldi[db][fy]
	L_{d1} &= \Ldi\\
	L_{d2} &= \Ldii\\
	d      &= \ddd \\
	h_z	   &= \hz
\end{align}

Para la altura de la zapata $h_z$, tomaremos el mayor valor de $L_{d1}$ y $L_{d2}$ más el recubrimiento.

\begin{align*}
    % Longitud de desarrollo
	% \Ldi[db][fy][f'c]
	% \Ldii[db][fy]
    L_{d1} &= \Ldi\\
	L_{d2} &= \Ldii\\
	d      &= \ddd \\
	h_z	   &= \hz
\end{align*}

\subsubsection{Capacidad portante neta del terreno}

El concepto de capacidad portante neta que es la capacidad del terreno reducida por efecto de la sobrecarga, el peso del suelo y el peso de la zapata. La capacidad portante neta es igual a:
\begin{align}
	\sigma_{sn} =&\qn \\
	\sigma_{sn} =&\qn[2400][0.60][2100][0.30][0.10][500]\\
	\sigma_{sn} =& valor-python \nonumber
\end{align}

\textbf{Donde:}

\begin{table}[h!]
    \centering
    \begin{tabular}{lll}
        $\sigma_{sn}$ &=&    Capacidad portante neta.\\
        $\sigma_t$ &=&   Carga admisible del terreno.\\
        $\gamma_c$ &=&    Peso específico del concreto\\
        $h_s$ &=&   Altura del suelo sobre la zapata.\\
    \end{tabular}
\end{table}

\newpage
\subsubsection{Dimensionamiento en planta}

\begin{theo}[Nota:]
    PPara el dimensionamiento en planta de una cimentación, considerar un incremento de la carga para tomar en cuenta el \textbf{peso propio de la zapata} (5 a 10\% dependiendo si el terreno es duro o blando)
    \begin{align}
        PP_{zapata} = 5-10\% (P_m + P_v)
    \end{align}
\end{theo}

Considerando peso de la zapata, calculamos el área tentativa.
\begin{align}
	A_t &= \AreaTentativa\\
	A_t &= 7.77 \, m^2\nonumber
\end{align}

Buscamos dos lados de zapata aproximadamente:
\begin{align}
	L &= \LongitudZapL \\
	B &= \LongitudZapB \\
	A &= \Area
\end{align}

Reemplazando valores:
\begin{align*}
	L &= \LongitudZapL[7.77][0.80][0.40] = 3.10\\
	B &= \LongitudZapB[7.77][0.80][0.40] = 2.70\\
	A &= \Area[3.10][2.70] = 
\end{align*}

\subsubsection{Verificación de zapatas con cargas y momentos biaxiales}

Si la carga aplicada viene acompañada con momentos que simultáneamente actúan en dos direcciones, asumiendo que la zapata es rígida y que la distribución de presiones sigue siendo lineal se puede obtener las presiones en las cuatro esquinas de una zapata rectangular con la siguiente expresión:
\begin{align}
	&\EcPresiones \\ \nonumber
	&\sigma_1 =\\ \nonumber
	&\sigma_2 =\\ \nonumber
	&\sigma_3 =\\ \nonumber
	&\sigma_4 =\\ \nonumber
\end{align}

Esta expresión será válida mientras no se tenga ninguna esquina con presión negativa,
lo cual implicaría admitir tracciones entre el suelo y la zapata.

\begin{figure}[H]
    \centering
    \includegraphics{images/VerPresiones.png}
    \caption{Verificación de presiones}
    \label{fig:my_label}
\end{figure}

\subsubsection{Diseño de la zapata}

Para el diseño por el método de resistencia o de cargas últimas, debemos amplificar las cargas según la combinación de cargas a usar.

Esto significa que deberíamos repetir todos los cálculos anteriores, amplificando las cargas y los momentos según las combinaciones indicadas por la norma, y obtener la presión última.

Sin embargo, este proceso puede ser simplificado, si amplificamos directamente la presión obtenida con cargas de servicio usando un coeficiente intermedio aproximado.

\begin{theo}[RNE - Norma E.060]
    LLa resistencia requerida para cargas muertas (CM) y cargas vivas (CV) será como mínimo:
    \begin{align}
        U = 1.4CM + 1.7CV
    \end{align}
    Si en el diseño se tuvieran que considerar cargas de sismo (CS), además de lo indicado en (5), la resistencia requerida será como mínimo:
    \begin{align}
        U &= 1.25(CM + CV) \pm CS\\
        U &= 0.90CM \pm CS
    \end{align}
\end{theo}


\begin{table}[h!]
    \centering
    \rowcolors{1}{}{gray!20}
    \begin{tabular}{ccc|c} \toprule
         \multicolumn{2}{c}{\textbf{Presión (kg/cm$^2$)}}   && \textbf{Presión de diseño (kg/cm$^2$)} \\ \midrule
        Sin carga sísmica   &   $\sigma_u =29.59 $  &&   $\sigma_u = 47.34$ \\ 
        Con sismo en X      &   $\sigma_u =34.31 $  &&   $\sigma_u = 42.89$ \\
        Con sismo en Y      &   $\sigma_u =34.17 $  &&   $\sigma_u = 42.71$ \\\bottomrule
    \end{tabular}
\end{table}

Por tanto se efectuará el diseño con $\sigma_u = 47.34$


\subsubsection{Verificación de corte por punzonamiento}
\begin{figure}[h!]
    \centering
    \includegraphics[width=0.75\textwidth]{images/DisPunzonamiento.png}
    \caption{Diseño por punzonamiento, a sección crítica se localiza a "d/2" de la cara}
    \label{fig:my_label}
\end{figure}
\begin{align}
	b_o &= \SeccCriticaP \\
	A_o &= \AreaTributariaP
\end{align}
\begin{align*}
	b_o &= \SeccCriticaP[0.80][0.40][0.50] \\
	A_o &= \AreaTributariaP[0.80][0.40][0.50]
\end{align*}

Cortante de diseño por punzonamiento:
\begin{align}
	V_u &= \VuPunzonamiento \\
	V_u &= \VuPunzonamiento[47.29][8.37][1.17]
\end{align}
\begin{align}
	V_u &= \VuPunzonamiento \\
	V_u &= \VuPunzonamiento[47.29][8.37][1.17]
\end{align}

Debe cumplirse que Vu $\leq$ $\phi$ Vc

Cortante resistente de concreto al punzonamiento:

\begin{theo}[Verificación del corte por punzonamiento]
    LLa resistencia del concreto al corte por punzonamiento es igual a la menor
determinada a través de las siguientes expresiones indicadas en la tabla 22.6.5.2
del ACI 318-14:
\end{theo}
\begin{align}
	\phi V_{c1} &= \VcPunzonamientoi \\
	\phi V_{c2} &= \VcPunzonamientoii \\
	\phi V_{c3} &= \VcPunzonamientoiii
\end{align}

\textbf{Donde:}

\begin{table}[h!]
    \centering
    \begin{tabular}{lll}
        $V_c$       &=& Resistencia del concreto al corte.\\
        $\beta$   &=& Cociente de la dimensión mayor de la columna entre la dimensión menor.\\
        $b_0$       &=& Perímetro de la sección crítica.\\
        $\alpha_s$  &=& \multirow{4}{14.6cm}{Parámetro igual a 40 para columnas interiores, 30 para las laterales y 20 para las esquineras. Se considera interiores aquellas en que la sección crítica de punzonamiento tiene 4 lados, laterales las que tienen 3 y esquineras las que tienen 2.}\\
                    & & \\
                    & & \\
                    & & \\
    \end{tabular}
\end{table}
Reemplazando valores:
\begin{align*}
	\phi V_{c1} &= \VcPunzonamientoi[][][][] \\
	\phi V_{c2} &= \VcPunzonamientoii[][][][][]\\
	\phi V_{c3} &= \VcPunzonamientoiii[][][]
\end{align*}

Como $Vu \leq  \phi V_c  $, cumple el valor de d=50 cm.Si no cumple aumentamos el peralte efectivo a
d + 10cm y volveremos a calcular.

\subsubsection{Verificación de corte por flexión}

\begin{figure}[H]
    \centering
    \includegraphics[width=0.75\textwidth]{images/DisCortante.png}
    \caption{Diseño por fuerza cortante}
    \label{fig:my_label}
\end{figure}

Cortante de diseño
\begin{align}
	V_u &= \VuFlexion \\
	V_u &= \VuFlexion[47.29][2.70][1.15][0.60] \nonumber \\
	V_u &=  70.2\nonumber
\end{align}

Cortante resistente
\begin{align}
	\phi V_{c} &= \VcFlexion \\
	\phi V_{c} &= \VcFlexion[f'_c][B][d] \nonumber
\end{align}

Debe cumplirse que Vu $\leq$ $\phi$ Vc

El peralte efectivo de 60 cm. es adecuado


\subsubsection{Cálculo de acero por flexión}
\begin{align}
	M_u &= \MomUltimoi \\
	M_u &= \MomUltimoi[47.30][1.15][2.70] \nonumber \\
	M_u &= \nonumber
\end{align}

Datos:
\[
\begin{array}{cc}
    M_u =   &   84.44   \,  kg\cdot m   \\
    \phi =  &   0.90                \\
    b =     &   270     \,  cm      \\
    d =     &   60      \,  cm      \\
    f'c =   &   210     \,  kg/cm^2
\end{array}
\]

\begin{align}
	&\MomUltimoii \\
	&\MomUltimoii[84.44][0.90][270][60][210] \nonumber
\end{align}

De la ecuación y los datos despejamos el valor de w.
\begin{align*}
    w_1 = \\
    w_2 = 
\end{align*}

Se toma el valor menor de w y se calcula el acero
\begin{align}
	A_s &= \AreaAcero \\
	A_s &= \AreaAcero[0.001][270][60][210][4200] \\ \nonumber
	A_s &= \nonumber
\end{align}

Acero mínimo
\begin{flalign}
	A_{s,min} &= \AreaAceroMin \\
	A_{s,min} &= \AreaAceroMin[270][50] \\\nonumber
	A_{s,min} &= \nonumber
\end{flalign}

\begin{theo}[De 9.7.3 y 10.5.4 del RNE - Norma E.060, (Acero mínimo)]
    NNos dice que, para zapatas de espesor uniforme el área mínima de acero para barras corrugadas  o malla de alambre (liso o corrugado) debe ser 0.0018 del área de la sección total de concreto.
\end{theo}

\clearpage
\bibliography{biblio}


