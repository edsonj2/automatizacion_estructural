\documentclass{article}%
\usepackage[T1]{fontenc}%
\usepackage[utf8]{inputenc}%
\usepackage{lmodern}%
\usepackage{textcomp}%
\usepackage{lastpage}%
\usepackage{geometry}%
\geometry{left=2.5cm,top=1.5cm}%
\usepackage[dvipsnames]{xcolor}%
\usepackage{graphicx}%
\usepackage{xargs}%
\usepackage{subfigure}%
\usepackage{array}%
\usepackage{multicol}%
\usepackage{multirow}%
\usepackage{fp}%
\usepackage{xcolor}%
\usepackage{booktabs}%
\usepackage{amsmath}%
%
%
%
\begin{document}%
\normalsize%
% Longitud de desarrollo %Varibles db, fy, f'c
\newcommandx{\Ldi}[3][1=d_b, 2=f_y, 3=f'_c]{0.08 \cdot #1 \cdot \dfrac{#2}{\sqrt{#3}}}
\newcommandx{\Ldii}[2][1=d_b, 2=f_y]{0.004 \cdot #1 \cdot #2}

% d = max(ld1, ld2) 
\newcommandx{\ddd}[2][1=L_{d1}, 2=L_{d2}]{\max(#1, #2)}

% hz = d + 0.1m
\newcommandx{\hz}[1][1=d]{d + 0.1m}

% Capacidad portante neta del terreno
% Variables: σn = σt - γc·hz - γm·hs - γc·hp - S/C_piso
\newcommandx{\qn}[7][1=\sigma_t, 2=\gamma_c, 3=h_z, 4=\gamma_m, 5=h_s, 6=h_p, 7=S/C_{piso}]{#1 - #2 \cdot #3 - #4\cdot#5 - #2\cdot#6 - #7}

% Área tentativa % Variables: Pservicio, σn
\newcommandx{\AreaTentativa}[2][1=P_{servicio}, 2=\sigma_n]{\dfrac{#1}{#2}}
 
% Dimensiones de la zapata L y B
% Variables: At, c1, c2
\newcommandx{\LongitudZapL}[3][1=A_t, 2=c_1, 3=c_2]{\sqrt{#1} + \dfrac{#2 - #3}{2}} %Para L
\newcommandx{\LongitudZapB}[3][1=A_t, 2=c_1, 3=c_2]{\sqrt{#1} - \dfrac{#2 - #3}{2}} %Para B
\newcommandx{\Area}[2][1=L, 2=B]{#1 \cdot #2}

% Verificación de zapatas con cargas y momentos biaxiales
% variables: Pultima, Area(A), Mx, Vx, Iyy + My, Vy, Ixx
\newcommandx{\EcPresiones}[8][1=P_{ultima}, 2=A, 3=M_x, 4=V_x, 5=I_{yy}, 6=M_y, 7=V_y, 8=I_{xx}]{\sigma_{1,2,3,4} = \dfrac{#1}{#2} \pm \dfrac{#3\cdot#4}{#5} \pm \dfrac{#6\cdot#7}{#8}}

% Verificación de corte por punzonamiento
% Sección crítica bo, variables: c1, c2, d
\newcommandx{\SeccCriticaP}[3][1=c_1, 2=c_2, 3=d]{2(#1 + #3) + 2(#2 + #3)}
% Area tributaria, variables: c1, c2, d
\newcommandx{\AreaTributariaP}[3][1=c_1, 2=c_2, 3=d]{(#1 + #3)\cdot(#2 + #3)}

% Cortante de diseño por corte-punzonamiento
% Variables: σu, A, Ao
\newcommandx{\VuPunzonamiento}[3][1=\sigma_u, 2=A, 3=A_o]{#1 \cdot (#2 - #3)}

% Verificación cortante por punzonamiento (las 3 verificaciones)
% Variables: σ=0.85, f'c, bo, d, αs, β
\newcommandx{\VcPunzonamientoi}[4][1=f'_c, 2=b_o, 3=d, 4=\beta]{0.85 \cdot (0.17 \left( 1 + \dfrac{2}{#4} \right) \sqrt{#1}\cdot #2 \cdot #3)}

\newcommandx{\VcPunzonamientoii}[5][1=f'_c, 2=b_o, 3=d, 4=\beta, 5=\alpha_s]{0.85 \cdot ( 0.083 \left( \dfrac{#5 \cdot #3}{#2} + 2\right) \sqrt{#1}\cdot #2 \cdot #3)}

\newcommandx{\VcPunzonamientoiii}[3][1=f'_c, 2=b_o, 3=d]{0.85 \cdot (0.33 \cdot \sqrt{#1} \cdot #2 \cdot #3)}


% Cortante de diseño por flexion
% Variables: σu, B, c, d
\newcommandx{\VuFlexion}[4][1=\sigma_u, 2=B, 3=c, 4=d]{#1 \cdot #2 \cdot (#3 - #4)}

% Verificación cortante por corte-flexion
% Variables: σ=0.85, f'c, B, d
\newcommandx{\VcFlexion}[3][1=f'_c, 2=B, 3=d]{0.85 \cdot 0.17 \cdot \sqrt{#1} \cdot #2 \cdot #3}

% Cálculo de acero por flexión
% Variables: σu, Lv=, B=prof de la zapata
\newcommandx{\MomUltimoi}[3][1=\sigma_u, 2=L_v, 3=B]{\dfrac{#1 \cdot (#2)^2 \cdot #3}{2}}

% Variables: Mu, φ, B, d, f'c, %%w
\newcommandx{\MomUltimoii}[5][1=M_u, 2=\phi, 3=B, 4=d, 5=f'c]{#1 = #2 (#3)(#4^2)(#5)(w)(1 - 0.59 \cdot w)}

% Cálculo del área de acero
% Variables: w, b, d, f'c, fy
\newcommandx{\AreaAcero}[5][1=w, 2=b, 3=d, 4=f'_c, 5=f_y]{\dfrac{#1 \cdot #2 \cdot #3 \cdot #4}{#5}}

% Área de acero mínimo
% Variables: b, h
\newcommandx{\AreaAceroMin}[2][1=b, 2=h]{0.0018 \cdot #1 \cdot #2}  
%
\section{Diseño de la Cimentación}%
\label{sec:DiseodelaCimentacin}%
\subsection{Diseño de Zapata Aislada}%
\label{subsec:DiseodeZapataAislada}%
\subsubsection{Datos para el diseño de una zapata aislada con carga y momentos}

\begin{table}[h!]
    \centering

    \begin{tabular}{lcl} %\toprule
        Dimensiones de la columna               &:& $C_1        = \hcol$m \quad $C_2 = \bcol$m\\
        Profundidad de cimentación              &:& $D_f        = 1.70 $m\\
        Altura de piso terminado                &:& $h_p        = 0.10$m\\
        Resistencia a compresión del concreto   &:& $f'_c       = 210$ kg/cm$^2$ \\
        Resistencia a la fluencia del acero     &:& $f_y        = 4200$ kg/cm$^2$ \\
        Peso específico del relleno             &:& $\gamma_m   = 2.1$ ton/m$^3$ \\
        Peso específico del concreto            &:& $\gamma_c   = 2.4$ ton/m$^3$ \\
        Sobrecarga de piso                      &:& $S/C_{piso}$= 500 kg/m$^2$ \\ 
        Capacidad portante del terreno          &:& $\sigma_t$  = 3 kg/cm$^2$  \\ %= 30 ton/m$^2$
        
    \end{tabular}
\end{table}

\textbf{Cargas:}

%
\subsubsection{Capacidad portante neta del terreno}

    El concepto de capacidad portante neta que es la capacidad del terreno reducida por efecto de la sobrecarga, el peso del suelo y el peso de la zapata. La capacidad portante neta es igual a:
    \begin{align}
        \sigma_{sn} =&\qn \\
        \sigma_{sn} =&\qn[1.20][2400.00][10.00][1400.00][100.00][40.00][100.00]\\
        \sigma_{sn} =& 0.93
    \end{align}

    \textbf{Donde:}

    \begin{table}[h!]
        \centering
        \begin{tabular}{lll}
            $\sigma_{sn}$ &=&    Capacidad portante neta.\\
            $\sigma_t$ &=&   Carga admisible del terreno.\\
            $\gamma_c$ &=&    Peso específico del concreto\\
            $h_s$ &=&   Altura del suelo sobre la zapata.\\
        \end{tabular}
    \end{table}
    %
\end{document}